\chapter{LANDASAN TEORI}
\section{Tinjauan Pustaka}
Dalam penelitian ini, beberapa referensi kepustakaan yang bersumber pada penelitian sebelumnya diambil sebagai bahan referensi. Referensi ini digunakan untuk membantu dalam menyelesaikan penelitian yang sedang dilakukan. Peneliti menggunakan referensi-referensi tersebut sebagai acuan dan sumber informasi untuk memperdalam pemahaman tentang topik yang sedang diteliti.

Pada penelitian terdahulu \textcite{suryadibrata2020visualisasi} membuat visualisasi algoritma \textit{K-Means Clustering} dalam bentuk 3D menggunakan unity dan menggunakan bahasa pemrograman C\char"0023\ yang dikembangkan dalam model \textit{Waterfall}. Aplikasi visualisasi ini bertujuan untuk membantu pelajar dalam memahami algoritma \textit{K-Means Clustering}. Aplikasi dalam penelitian ini mengimplementasikan animasi dari proses perhitungan jarak antara setiap vektor data dan setiap vektor pusat yang ada menggunakan sebuah garis pembantu. Dalam proses visualisasi garis pembantu berperan untuk menunjukkan proses pencarian jarak terdekat dari setiap vektor data. Berdasarkan latar belakang bahwa banyak pelajar kesulitan memahami algoritma pembelajaran mesin karena kompleksitas matematisnya, penelitian ini menggunakan metode visualisasi algoritma untuk menjelaskan langkah-langkah K-Means Clustering secara interaktif. Hasil penelitian menunjukkan bahwa penggunaan elemen grafis seperti titik dan garis, serta animasi perubahan warna vektor data, membantu pelajar lebih mudah memahami proses pengelompokan data. Kesimpulannya, visualisasi algoritma K-Means Clustering ini dapat meningkatkan motivasi dan pemahaman pelajar terhadap materi yang kompleks.

Dalam penelitian kedua yang dilakukan oleh \textcite{zarkasyi2015pengembangan}, penggunaan aplikasi GeoGebra untuk media pembelajaran bertujuan untuk memvisualisasikan penggunaan integral pada siswa SMA. Kriteria kualitas media pembelajaran mencakup aspek kualitas visual dan teknis. Penelitian ini menggunakan model pengembangan ADDIE (\textit{Analysis, Design, Development, Implementation, Evaluation}). Latar belakang masalah adalah kesulitan siswa dalam memahami konsep abstrak penggunaan integral untuk menghitung luas daerah di bawah kurva dan volume benda putar. Metode pengembangan ADDIE dimulai dengan analisis kebutuhan dan karakteristik siswa, desain media, pengembangan media dengan GeoGebra, implementasi kepada guru dan siswa, serta evaluasi kualitas produk. Hasil penelitian menunjukkan bahwa media pembelajaran yang dikembangkan efektif dan layak digunakan dengan kualitas yang baik. Kesimpulan dari penelitian ini adalah bahwa penggunaan media pembelajaran berbasis GeoGebra dapat membantu siswa memvisualisasikan konsep integral dan meningkatkan pemahaman mereka terhadap materi tersebut

Berdasarkan tinjauan pustaka yang telah disajikan, dapat disimpulkan bahwa penggunaan aplikasi visualisasi dan media pembelajaran interaktif dapat meningkatkan kualitas pembelajaran serta motivasi belajar mahasiswa. Penelitian terdahulu yang telah dilakukan oleh para peneliti sebelumnya dapat menjadi acuan dan bahan pertimbangan dalam menganalisis hasil penelitian ini. Diharapkan penelitian ini dapat memberikan sumbangsih bagi pengembangan media pembelajaran yang lebih inovatif dan efektif untuk meningkatkan kualitas pendidikan.

Penelitian ini memiliki kesamaan dengan penelitian sebelumnya yaitu keduanya menggunakan aplikasi visualisasi sebagai media pembelajaran. Namun, terdapat perbedaan dimana penelitian sebelumnya memvisualisasikan K-Means Clustering dengan menggunakan bahasa pemrograman C\char"0023\ berbasis aplikasi \textit{desktop}, sedangkan penelitian ini memvisualisasikan operasi pengolahan citra digital dengan menggunakan bahasa pemrograman \textit{JavaScript} berbasis \textit{Web}.

\section{Dasar Teori}
\subsection{Pengolahan Citra Digital}
\textcite{munantri2020aplikasi} menjelaskan bahwa citra digital adalah gambar dua dimensi yang dihasilkan dari proses sampling pada gambar analog dua dimensi yang kontinu, sehingga menjadi gambar diskrit yang dapat diolah oleh komputer. Citra digital disimpan dalam bentuk data numerik yang menunjukkan besar intensitas pada masing-masing pikselnya. Oleh karena itu, citra digital dapat diolah dengan menggunakan komputer untuk berbagai keperluan seperti pengolahan gambar, analisis citra, dan pengenalan pola.

Masih menurut \textcite{munantri2020aplikasi} menjelaskan bahwa pengolahan citra digital merupakan ilmu yang mempelajari berbagai hal terkait dengan perbaikan kualitas gambar, transformasi gambar, pemilihan citra ciri yang optimal, penyimpanan data, transmisi data, dan waktu proses data. Proses pengolahan citra dapat dijelaskan dengan menggunakan diagram sederhana yang meliputi beberapa tahap, seperti perbaikan kualitas citra, transformasi citra, pemilihan citra ciri, reduksi dan kompresi data, transmisi data, dan waktu proses data. Dalam pengolahan citra digital, teknik-teknik pengolahan yang digunakan dapat berbeda-beda tergantung pada jenis citra dan tujuan analisis yang ingin dicapai.

Dari uraian diatas penulis menyimpulkan bahwa citra digital adalah hasil dari proses sampling pada gambar analog dua dimensi yang kemudian diubah menjadi gambar diskrit dengan data numerik yang merepresentasikan besar intensitas pada masing-masing piksel. Citra digital dapat diolah dengan menggunakan komputer dan teknik-teknik pengolahan citra yang berbeda-beda tergantung pada jenis citra dan tujuan analisis yang ingin dicapai. Proses pengolahan citra digital meliputi berbagai tahap seperti perbaikan kualitas citra, transformasi citra, pemilihan citra ciri, reduksi dan kompresi data, transmisi data, dan waktu proses data. Dalam ilmu pengolahan citra digital, terdapat berbagai macam teknik yang dapat digunakan untuk mengambil informasi dari citra digital atau memperbaiki kualitas citra.

\begin{enumerate}[leftmargin=1cm, itemindent=0.6cm,labelwidth=15pt, labelsep=5pt, listparindent=1cm,align=left]

\item Operasi \textit{Grayscale}

    Operasi \textit{grayscale} adalah proses normalisasi tiga lapisan warna RGB dari sebuah citra berwarna menjadi satu lapisan grayscale \cite{suryowinoto2017penggunaan}. Dalam citra digital, grayscale memiliki gradasi warna dari putih ke hitam, seperti yang ditunjukkan oleh rumus (2.1). Rentang ini menunjukkan bahwa setiap piksel direpresentasikan oleh 8 bit. Karena citra digital grayscale sebenarnya adalah hasil rata-rata (dinormalisasi) dari citra berwarna, persamaannya dapat dituliskan sebagai berikut:

\begin{equation}
f_o(x,y) = \frac{f_i^R(x,y) + f_i^G(x,y) + f_i^B(x,y)}{3}
\end{equation}

Di mana \(f_i^R(x,y)\) adalah nilai piksel warna merah pada titik \((x,y)\), \(f_i^G(x,y) \) adalah nilai piksel warna hijau pada titik \((x,y)\), dan \(f_i^B(x,y)\) adalah nilai piksel warna biru pada titik \((x,y)\).

\item Operasi \textit{Invert}

    Citra invert, atau citra negatif, adalah citra yang merupakan kebalikan dari citra asli \cite{mahardika2017implementasi}. Mirip dengan film negatif yang dihasilkan dari kamera konvensional. Jika sebuah citra memiliki jumlah \textit{gray level} L dengan rentang dari 0 hingga L-1, citra negatif dapat diperoleh melalui transformasi negatif yang dijelaskan oleh persamaan berikut:

\begin{equation}
s = L - 1 - r
\end{equation}

Di mana:
s = citra hasil transformasi negatif\\
L = jumlah \textit{gray level} sebuah citra\\
r = citra asli

	\item Operasi \textit{Brightness}

        Proses operasi \textit{brightness} dilakukan dengan menambahkan atau mengurangkan nilai setiap piksel dengan suatu konstanta \cite{SPEKTRUM}. Penyesuaian tingkat kecerahan suatu citra dapat dinyatakan
sebagai:

        \begin{equation}
            U = U + c
        \end{equation}

dengan U dan U berturut-turut menyatakan citra setelah dan sebelum brightness adjustment sedangkan c adalah suatu konstanta yang merupakan variabel penyesuaian.

\item Operasi \textit{Threshold}

    \textit{Thresholding} adalah metode segmentasi citra yang memisahkan objek dari latar belakang berdasarkan perbedaan kecerahan \cite{Setiawan_Dewanta_Nugroho_Supriyono_2019}. Daerah yang lebih gelap akan semakin gelap (hitam sempurna dengan nilai intensitas 0), sedangkan daerah yang lebih terang akan semakin terang. Operasi ambang batas tunggal membagi nilai piksel menjadi dua kelompok seperti ditunjukkan oleh rumus berikut:

        \begin{equation}
            g(x,y) = 
            \left\{
            \begin{array}{ll}
            0, & \text{jika } f(x,y) < T \\
            255, & \text{jika } f(x,y) \geq T 
            \end{array}
            \right \}
        \end{equation}

Piksel dengan nilai intensitas di bawah T akan diubah menjadi hitam (nilai intensitas 0), sedangkan piksel dengan nilai intensitas di atas T akan diubah menjadi putih (nilai intensitas 255).

\item Operasi \textit{Image Blending}

Image blending atau penjumlahan citra adalah teknik menggabungkan dua atau lebih citra untuk menghasilkan citra baru. Teknik ini sering digunakan dalam aplikasi seperti penggabungan panorama, pengurangan noise, dan efek khusus. Salah satu metode umum adalah blending linear, di mana dua citra digabungkan dengan bobot tertentu. Jika ada dua citra \(I_1\) dan \(I_2\), rumusnya adalah:

\begin{equation}
I_{\text{blended}}(x, y) = \alpha \cdot I_1(x, y) + \beta \cdot I_2(x, y)
\end{equation}

Di mana \(I_1(x, y)\) dan \(I_2(x, y)\) adalah intensitas piksel pada posisi \((x,y)\) dari citra pertama dan kedua, sedangkan \(\alpha\) dan \(\beta\) adalah bobot untuk masing-masing citra, dengan \(\alpha + \beta = 1\).

\item Operasi \textit{Image Subtraction}

Image subtraction adalah teknik pengolahan citra yang melibatkan pengurangan nilai intensitas piksel dari satu citra dengan nilai intensitas piksel yang bersesuaian dari citra lain. Teknik ini sering digunakan untuk mendeteksi perubahan atau mengurangi latar belakang. Jika ada dua citra \(I_1\) dan \(I_2\), pengurangan citra dapat dinyatakan sebagai:

\begin{equation}
I_{\text{subtracted}}(x, y) = I_1(x, y) - I_2(x, y)
\end{equation}

Di mana \(I_{\text{subtracted}}(x, y)\) adalah intensitas piksel pada posisi \((x,y)\) dari citra hasil pengurangan, \(I_1(x, y)\) adalah intensitas piksel pada posisi \((x,y)\) dari citra pertama, dan \(I_2(x, y)\) adalah intensitas piksel pada posisi \((x,y)\) dari citra kedua. Untuk mengatasi nilai negatif, bisa menambahkan nilai konstan yang cukup besar untuk menggeser semua nilai ke rentang positif, misalnya dengan menambahkan 255 ke setiap nilai hasil pengurangan, lalu melakukan clipping untuk memastikan nilai tidak melebihi batas atas (255) sebagai berikut:

\begin{equation}
I_{\text{normalized}}(x, y) = \max(0, \min(255, I_{\text{subtracted}}(x, y) + 255))
\end{equation}


\end{enumerate}
\subsection{Media Pembelajaran}
Menurut \textcite{arsyad2015media} kata media berasal dari bahasa latin \textit{medius} yang secara harfiah berarti ``tengah'', ``perantara'' atau ``pengantar'', yang dalam bahasa arab diartikan sebagai perantara atau pengantar pesan dari pengirim kepada penerima pesan. Oleh karena itu, media diartikan sebagai alat yang menyampaikan atau mengantarkan pesan-pesan pengajaran.

\textcite{nurrita2018pengembangan} menjelaskan bahwa media pembelajaran berperan sebagai alat yang dapat membantu proses belajar mengajar sehingga makna pesan yang disampaikan dapat menjadi lebih jelas dan tujuan pendidikan atau pembelajaran dapat tercapai dengan efektif dan efisien.

Berdasarkan uraian diatas dapat disimpulkan bahwa media pembelajaran merupakan alat atau sarana yang berperan sebagai perantara atau pengantar pesan dalam proses belajar mengajar. Dalam proses pembelajaran, media pembelajaran dapat membantu membuat pesan yang disampaikan menjadi lebih jelas dan tujuan pembelajaran dapat tercapai dengan efektif dan efisien. Oleh karena itu, penggunaan media pembelajaran sangat penting dalam proses belajar mengajar untuk mencapai hasil yang optimal.

\subsection{Visualisasi}
Visualisasi memiliki beberapa landasan teori yang mendasar, di antaranya adalah teori pengolahan informasi, teori persepsi visual, dan teori kognitif. Teori pengolahan informasi menjelaskan bahwa manusia memproses informasi dengan cara menerima, menyimpan, mengorganisir, dan mengambil kembali informasi dalam memori. Teori ini menjadi dasar bagi visualisasi karena visualisasi digunakan untuk mempermudah pengolahan informasi dengan menggambarkan data dalam bentuk visual yang mudah dipahami.

Teori persepsi visual menjelaskan tentang bagaimana manusia mengamati dan menginterpretasikan dunia visual. Teori ini menjelaskan bagaimana manusia memahami bentuk, warna, ukuran, dan posisi objek dalam lingkungan visual. Teori ini menjadi penting dalam visualisasi karena visualisasi berupaya untuk membuat gambaran yang tepat dan mudah dipahami oleh pengamat.

Sedangkan teori kognitif membahas tentang bagaimana manusia memproses, menyimpan, mengambil, dan menggunakan informasi yang telah dipelajari. Teori ini menjadi dasar bagi pengembangan visualisasi karena visualisasi bertujuan untuk meningkatkan pemahaman dan retensi informasi pada manusia dengan menyajikan informasi dalam bentuk yang mudah diingat dan dipahami.

Dari ketiga landasan teori di atas, dapat disimpulkan bahwa visualisasi sebagai alat bantu memahami informasi berusaha mempermudah pengolahan informasi manusia dengan menggunakan teori pengolahan informasi, teori persepsi visual, dan teori kognitif sebagai dasar pengembangan.

\subsection{Web}
Web atau \textit{World Wide Web} (WWW) adalah sistem informasi global yang terhubung melalui jaringan internet dan digunakan untuk mengakses dan berbagi informasi di seluruh dunia. Web pertama kali diperkenalkan pada tahun 1989 oleh Tim Berners-Lee, seorang ilmuwan komputer dari Inggris. Dalam pengertian umum, web adalah kumpulan dokumen atau halaman web yang terdiri dari teks, gambar, video, dan berbagai jenis konten multimedia lainnya.

Pada awalnya, web hanya digunakan sebagai alat untuk membagikan informasi dan sebagai tempat untuk mengakses situs web. Namun, seiring dengan perkembangan teknologi, web kini telah menjadi lebih interaktif dan dinamis. Contohnya adalah adanya aplikasi web yang memungkinkan pengguna untuk melakukan transaksi online, berinteraksi dengan orang lain, bermain game, bahkan menjadi media pembelajaran.

Web sendiri terdiri dari tiga komponen utama, yaitu bahasa markup (HTML), style sheet language (CSS), dan bahasa scripting (JavaScript). HTML digunakan untuk membuat struktur dasar halaman web, sedangkan CSS digunakan untuk mendesain tampilan halaman web, dan JavaScript digunakan untuk membuat halaman web menjadi interaktif.

\subsection{Alat Bantu Pengembangan di Perancangan Sistem}

Adapun alat bantu untuk pengembangan perancangan sistem aplikasi media pembelajaran ini adalah

\begin{enumerate}[leftmargin=1cm, itemindent=0.6cm,labelwidth=15pt, labelsep=5pt, listparindent=1cm,align=left]

    \item \textit{Activity Diagram}

        \textit{Activity diagram} adalah salah satu jenis diagram dalam Unified Modeling Language (UML) yang digunakan untuk memodelkan alur kerja atau proses bisnis dalam sebuah sistem. \textit{Activity diagram} menggambarkan berbagai alur kegiatan dalam sistem yang sedang dirancang, mulai dari kondisi awal setiap alur, keputusan yang mungkin diambil, hingga akhir dari kegiatan tersebut \cite{pratama2021rancang}.Diagram ini memberikan representasi visual dari aktivitas dan urutan tindakan yang terjadi dalam sebuah proses, menunjukkan bagaimana satu aktivitas mengarah ke aktivitas lainnya. Dengan kata lain, activity diagram menggambarkan aliran kontrol dari satu aktivitas ke aktivitas lain dan bagaimana berbagai kondisi dan keputusan mempengaruhi aliran ini.

        \textit{Activity diagram} berguna dalam menggambarkan proses yang kompleks, memperjelas logika alur kerja, dan mengidentifikasi potensi perbaikan atau optimasi. Diagram ini terdiri dari elemen-elemen seperti \textit{activity nodes} (aktivitas), \textit{control flows} (aliran kontrol), \textit{decision nodes} (simpul keputusan), dan \textit{merge nodes} (simpul penggabungan). Elemen-elemen ini memungkinkan pengembang dan pemangku kepentingan untuk memahami proses secara menyeluruh, memfasilitasi komunikasi yang efektif, dan memastikan bahwa semua aspek dari alur kerja telah dipertimbangkan.

	\item Flowchart

	      menyatakan bahwa desain arsitektur program memiliki peran penting dalam menentukan hubungan antara elemen-elemen struktural utama dalam program. Desain arsitektur dapat dijabarkan dalam bentuk diagram alir program atau flowchart, yang menggunakan simbol-simbol khusus untuk menyatakan aliran proses program dan alur proses yang dikehendaki.

	      Diagram alir program terdiri dari beberapa simbol yang masing-masing memiliki arti dan fungsi yang berbeda-beda. Beberapa simbol umum yang sering digunakan dalam flowchart antara lain:

	      \begin{enumerate}
		      \item Oval: Menunjukkan awal atau akhir dari proses.
		      \item Kotak: Menunjukkan proses yang harus dilakukan
		      \item Panah: Menunjukkan arah aliran proses
		      \item Diamond: Menunjukkan kondisi atau percabangan dalam proses
	      \end{enumerate}

	      Dalam membuat flowchart, perlu diperhatikan juga urutan proses yang logis dan jelas, sehingga dapat memudahkan penggunaan flowchart dalam memahami alur proses program. Selain itu, flowchart juga perlu disesuaikan dengan tujuan dan kebutuhan penggunaannya.

\end{enumerate}

\subsection{\textit{Software} Pendukung}
\begin{enumerate}[leftmargin=1cm, itemindent=0.6cm,labelwidth=15pt, labelsep=5pt, listparindent=1cm,align=left]

	\item JavaScript

	      JavaScript adalah bahasa pemrograman yang sangat penting dalam pembuatan website dan aplikasi web. Bahasa pemrograman ini sering digunakan untuk menambahkan interaktifitas pada website, membuat efek animasi, memvalidasi input pengguna, dan banyak lagi. Dalam konteks pengolahan citra digital menggunakan visualisasi interaktif berbasis web, JavaScript dapat digunakan untuk membuat tampilan interaktif yang memudahkan pengguna dalam memahami proses pengolahan citra yang sedang dilakukan.

	      Salah satu contoh penggunaan JavaScript dalam pembuatan media pembelajaran pengolahan citra digital adalah dengan membuat animasi yang menggambarkan proses sampling pada citra analog dan proses konversi menjadi citra digital. Dalam animasi ini, JavaScript dapat digunakan untuk mengatur waktu tampilan dan memanipulasi objek-objek visual seperti gambar, teks, dan grafik. Selain itu, JavaScript juga dapat digunakan untuk membuat interaksi pengguna yang memungkinkan mereka untuk memilih dan memanipulasi citra digital dengan mudah.

	      Penggunaan JavaScript dalam media pembelajaran pengolahan citra digital juga dapat diperluas untuk mencakup fitur-fitur lain seperti pengolahan filter dan perubahan kontras pada citra digital. Dengan menggunakan JavaScript, pengguna dapat dengan mudah memilih filter yang ingin digunakan dan melihat perbedaan sebelum dan sesudah filter diterapkan pada citra digital. Selain itu, JavaScript juga dapat digunakan untuk menampilkan grafik dan diagram yang membantu pengguna dalam memahami data citra digital yang kompleks.

	      Dalam keseluruhan, penggunaan JavaScript dalam media pembelajaran pengolahan citra digital dengan visualisasi interaktif berbasis web sangat penting dalam meningkatkan interaktivitas dan pemahaman pengguna terhadap proses pengolahan citra. Dengan penggunaan teknologi yang tepat dan penanganan dengan hati-hati, media pembelajaran ini dapat menjadi alat yang efektif untuk memperkenalkan dan mengajarkan konsep-konsep pengolahan citra digital kepada mahasiswa dan pembelajar di mana saja dan kapan saja.

	\item Bootstrap

	      Bootstrap adalah salah satu \textit{framework} CSS (\textit{Cascading Style Sheets}) yang populer dan digunakan secara luas untuk membangun tampilan website yang responsif dan menarik. \textit{Framework} ini memiliki banyak komponen UI (\textit{User Interface}) dan JS (JavaScript) yang siap pakai, sehingga mempermudah proses pengembangan website.

	      Dalam konteks ini penulis menggunakan Bootstrap yang dapat mempercepat proses pengembangan tampilan website. Framework ini menyediakan komponen UI yang dapat di-\textit{customize} dan disesuaikan dengan kebutuhan, seperti \textit{grid system, typography, form, button}, dan lain-lain. Dalam pembuatan website media pembelajaran yang interaktif, Bootstrap juga menyediakan komponen JS yang dapat digunakan, seperti \textit{modal, collapse, dan carousel}.

	      Keunggulan lain dari Bootstrap adalah dukungannya terhadap desain responsif atau \textit{mobile-friendly}. Dalam era digital yang semakin mobile, penggunaan Bootstrap dapat memastikan tampilan website dapat menyesuaikan dengan ukuran layar yang berbeda-beda, sehingga pengalaman pengguna yang baik dapat tetap terjaga.

	      Secara keseluruhan, Bootstrap dapat menjadi pilihan yang baik dalam pengembangan tampilan website media pembelajaran pengolahan citra digital yang interaktif dan responsif. Dengan banyaknya komponen UI dan JS yang tersedia, pengembangan website dapat menjadi lebih cepat dan efisien.
\end{enumerate}
